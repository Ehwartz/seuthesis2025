\documentclass{seuthesis2025}
% \documentclass{seuthesis2024b} % 使用 fonts 文件夹字体
% \documentclass[twoside]{seuthesis2024b} % 使用双页模式
% 更多自定义选项见 seuthesis2024b.cls 文件的定义
% 注意选项可以合并使用
\usepackage{tikz}
\usetikzlibrary{calc} % 如果你要做位置计算,也可以加
\usetikzlibrary{backgrounds}
\usepackage{eso-pic} % 必须的,用于在页面背景上叠加内容

\title{\hspace{3em}东南大学本科毕设 \LaTeX{} 模版}

% \title{东南大学本科毕设 \LaTeX{} 模版}[第二行内容]
% \title{东南大学本科毕设 \LaTeX{} 模版}[第二行内容写满了还是不太塞得下啊][-2] % 此时使用 \zihao{-2} 大小
\studentID{12345678}
\author{泰迪熊}
\department{泰迪熊学院}
\major{毛茸茸工程}
\supervisor{Ted Roosevelt}
\date{2023年12月 至 2024年5月}

% % default arXiv date is 2024-03-01
% \makeatletter
% \renewcommand \seuthesis@arxiv@urldate {2024-04-01} % the date you want
% \makeatother

% 添加参考文献 bib 文件,注意使用符合 BibLaTeX(配合 Biber)的格式
\addbibresource{ref/IEEEfull.bib}
\addbibresource{ref/ref.bib}

% amsthm 定理设置
\newtheorem{theorem}{定理}[chapter]
\newtheorem{definition}{定义}[chapter]
\newtheorem{lemma}{引理}[chapter]
\newtheorem{corollary}{推论}[chapter]

% 中文摘要和关键词
\CNabstract{%
  此 \LaTeX{} 模板基于东南大学教务处 2024 年 1 月最新 MS Word 版本制作。

  摘要内容独立于正文而存在,是论文内容高度概括的简要陈述, 
  应准确、具体、完整地概括论文的主要信息,内容包括研究目的、方法、过程、成果、结论及主要创新之处等,
  不含图表,不加注释,具有独立性和完整性,一般为400字左右。

  “摘要”用三号黑体加粗居中,“摘”与“要”之间空4个半角空格。摘要正文内容用小四号宋体,固定1.5倍行距。

  论文的关键词是反映毕业设计(论文)主题内容的名词,一般为3-5个,排在摘要正文部分下方。关键词与摘要之间空一行。
  关键词之间用逗号分开,最后一个关键词后不加标点符号。
} {关键词1,关键词2,关键词3,关键词4}

% 英文摘要和关键词
\ENabstract{%
  English abstract should correspond to the contents in the Chinese abstract.
  It should describe the thesis contents as a third-person perspective.
  The basic tense used in the abstract is the present tense.
  Other tenses like the past tense or the completion tense should only be used when necessary.

  ABSTRACT should be centered and bolded with Times New Roman \texttt{\string\zihao\{3\}}.

  The English abstract contents format is the same as the Chinese abstract.
  The English ``KEY WORDS'' should be in all uppercase,
  and all keywords have the first letter in capital.
  Keywords are separated by an English comma.
} {Keywords 1, Keywords 2, Keywords 3, Keywords 4}

\begin{document}
  \maketitle

  \chapter{绪论}\label{chap:introduction}
  \section{课题背景和意义}
  绪论部分主要论述选题的意义、国内外研究现状以及本文主要研究的内容、研究思路以及内容安排等。

  章标题为三号黑体加粗居中;一级节标题(如,2.1 本文研究内容):四号黑体居左;二级节标题(如,2.1.1 实验方法):小四号宋体居左。

  正文部分为小四号宋体,行间距1.5倍行距,首行缩进2个字符。正文一般不少于15000字。

\section{研究现状}
  ……

\section{本文研究内容}
  ……

    
  \chapter{正文}\label{chap:body}
  具体研究内容每一章应另起页书写书写,层次要清楚,内容要有逻辑性,每一章标题需要按论文实际研究内容进行填写,不可直接写成第二章 正文。研究内容因学科、选题特点可有差异,但必须言之成理,论据可靠,严格遵循本学科国际通行的学术规范。

中文为小四号宋体,英文及数字为小四号Times New Roman,首行缩进2个字符,行间距为1.5倍。

\section{插图格式要求}

  插图力求精炼,且每个插图均应有图序和图名。图序与图名位于插图下方,图序一般按章节编排,如图1-1(第一章第1个图),在插图较少时可以全文连续编序,如图10。
  
  如一个插图由两个及以上的分图组成,分图用(a)、(b)、(c)等标出,并标出分图名。

  简单文字图可用 \LaTeX{} 自带宏包 Ti\textit{k}Z 直接绘制,
  复杂的图考虑使用 standalone 的 Ti\textit{k}Z 完成,以提高图形表达质量。

  插图居中排列,与上文文本之间空一行。图序图名设置为五号宋体居中,图序与图名之间空一格。
  可爱的图~\ref{fig:duck}。
  \begin{figure}[htbp]
    \includegraphics{example-image-duck}
    \caption{经典 \LaTeX{} 鸭子}
    \label{fig:duck}
  \end{figure}

  图~\ref{fig:subfigs} 包含两个子图:
  图~\subref*{subfig:s1} 和 \subref*{subfig:s2}。
  \begin{figure}[htbp]
    \subfloat[子图一\label{subfig:s1}]{\includegraphics[width=.3\linewidth]{example-image-a}}\quad
    \subfloat[子图二\label{subfig:s2}]{\includegraphics[width=.3\linewidth]{example-image-b}}
    \caption{两个子图}
    \label{fig:subfigs}
  \end{figure}

  图序章节编排为默认,全文连续编排需设置文档选项 \texttt{cnt in doc}。此设置同时作用于表格和公式。

\section{表格格式要求}

  表格的结构应简洁,一律采用三线表,应有表序和表名,且表序和表名位于表格上方。
  表格可以逐章单独编序(如:表2.1),也可以统一编序(如:表10),采用哪种方式应和插图及公式的编序方式统一。表序必须连续,不得重复或跳跃。

  表格无法在同一页排版时,可以用续表的形式另页书写,续表需在表格右上角表序前加“续”字,如“续表2.1”,并重复表头。

  表格居中,边框为黑色直线1磅,中文为五号宋体,英文及数字为五号Times New Roman字体,表序与表名之间空一格,表格与下文之间空一行。

  表格例子如表~\ref{tab:font_effect} 所示。
  表格内字号默认为五号(10.5pt),即 \texttt{\string\small} 或 \texttt{\string\zihao\{5\}}。

  \begin{table}[htbp]
    \caption{字体族、字体形状和字体系列的组合效果}
    \label{tab:font_effect}
    \begin{tabular}{ccccc}
      \toprule
      & & \verb|\itshape| & \verb|\bfseries| & \verb|\itshape\bfseries| \\
      \midrule
      \verb|\rmfamily| & \rmfamily 罗马体 roman & \rmfamily\itshape 倾斜 italic & \rmfamily\bfseries 加粗 bold & \rmfamily\itshape\bfseries 粗斜 bold-italic \\
      \verb|\sffamily| & \sffamily 无衬线 sans  & \sffamily\itshape 倾斜 italic & \sffamily\bfseries 加粗 bold & \sffamily\itshape\bfseries 粗斜 bold-italic \\
      \verb|\ttfamily| & \ttfamily 打字机 mono  & \ttfamily\itshape 倾斜 italic & \ttfamily\bfseries 加粗 bold & \ttfamily\itshape\bfseries 粗斜 bold-italic \\
      \bottomrule
    \end{tabular}
  \end{table}

  跨页表格可以使用 \texttt{longtable} 环境。
  但是为了更好的表格居中效果,推荐使用 \texttt{xltabular} 环境,
  将 \texttt{longtable} 与 \texttt{tabularx} 结合。
  效果如表~\ref{tab:arabic_roman}。

  \begin{xltabular}{\linewidth}{YYYY} % <-- centered X columns
    \caption{阿拉伯数字与罗马数字转换表} \label{tab:arabic_roman} \\
    \longTableHdr{阿拉伯数字 & 罗马数字 & 阿拉伯数字 & 罗马数字 \\}
    1 & \RN{1} & 2 & \RN{2} \\
    3 & \RN{3} & 4 & \RN{4} \\
    5 & \RN{5} & 6 & \RN{6} \\
    7 & \RN{7} & 8 & \RN{8} \\
    9 & \RN{9} & 10 & \RN{10} \\
    11 & \RN{11} & 12 & \RN{12} \\
    13 & \RN{13} & 14 & \RN{14} \\
    15 & \RN{15} & 16 & \RN{16} \\
    17 & \RN{17} & 18 & \RN{18} \\
    19 & \RN{19} & 20 & \RN{20} \\
    50 & \RN{50} & 100 & \RN{100} \\
    500 & \RN{500} & 2024 & \RN{2024} \\
  \end{xltabular}

\section{表达式}

  论文中的公式应注序号并加圆括号,序号一律用阿拉伯数字连续编序(如 (28) )或逐章编序(如 (3.6) ),
  编号方式应与插图、表格方式一致。序号排在版面右侧,且距右边距离相等。公式与序号之间不加虚线。

  长公式在一行无法写完的情况下,原则上应在等号(或数学符号,如“$+$”、“$-$”号)处换行,数学符号在换行的行首。

  公式及文字中的一般变量(或一般函数)(如坐标$X$、$Y$,电压$V$,频率$f$)宜用斜体,矢量用粗斜体如$\bm{S}$或白斜体上加单箭头,
  常用函数(如三角函数$\cos$、对数函数$\log$等)、数字运算符、化学元素符号及分子式、单位符号、产品代号、人名地名的外文字母等用正体。

  公式排版直接使用 \texttt{equation} 环境,如公式~\eqref{eq:einstein}。
  \begin{equation}\label{eq:einstein}
    E=mc^2.
  \end{equation}

\section{注释}

  正文中有个别名词或情况需要解释时,可加注说明,注释采用页末注(将注文放在加注页的下端)。
  在引文的右上角标注序号 \circNo{1}、\circNo{2}、……,如“马尔可夫链\footnote{马尔可夫链表示……}”。若在同一页中有两个以上的注时,按各注出现的先后,顺序编号。
  引文序号,以页为单位,且注释只限于写在注释符号出现的同页,不得隔页。

  注释采用小五号宋体,英文及数字为小五号Times New Roman字体,利用 \texttt{\string\footnote} 插入。

\section{参考文献}

  \subsection{引用格式}
    列出作者直接阅读过或在正文中引用过的文献资料。撰写论文时,需注意引用权威和最新的文献。

    参考文献需在引文右上角用方括号“[]”标明序号,如“基本机构\textsuperscript{[1]}”,并在参考文献中列出。
    每一条参考文献著录均以“.”结束。参考文献要另起一页,一律放在正文之后,不得放在各章节之后。

    参考文献采用顺序编码制,需符合《信息与文献 参考文献著录规则》(GB/T 7714-2015)规范要求,文献类型和标识代码为:
    普通图书[M]、会议录[C]、汇编[G]、报纸[N]、期刊[J]、学位论文[D]、报告[R]、标准[S]、专利[P]、数据库[DB]、
    计算机程序[CP]、电子公告[EB]、档案[A]、舆图[CM]、数据集[DS]、其他[Z]。

    参考文献中主要责任者的个人作者采用姓在前名在后的著录形式,当作者不超过3个时,全部照录。
    超过3个,著录的前3个作者其后加“,等”(, et al)。
    欧美著者的名可用缩写字母,缩写名后省略缩写点,姓和缩写名全大写。
    用汉语拼音书写的人名,姓全大写,名可缩写,取每个汉字拼音的首字母。

    参考文献为五号宋体,英文及数字为五号Times New Roman字体,两端对齐。
    参考文献中的标点符号均为英文标点,常用的参考文献著录项目和格式见\hyperref[chap:bib]{参考文献}%
    和第~\ref{subsec:bib_types} 节。

  \subsection{使用示例}
    算法方面的研究包括:
    OMPL-SBL 算法\cite{zhao2023ompl},
    RIS 的波束设计方法\cite{you2024beam};
    硬件设计方面的研究包括:
    高层级综合工具 FLAMES 库设计\cite{zhao2024flexible},
    自动生成语言 AHDW\cite{zhao2023automatic}。
    \name 的所有工作有\cite{zhao2023ompl,you2024beam,zhao2024flexible,zhao2023automatic}。

    使用 \texttt{\string\cite} 的引用均为上标;
    如果需要非上标,可以使用 \texttt{\string\parencite},例如 \parencite{zhao2023ompl}。
    此模版额外提供了 \texttt{\string\Lcite} 和 \texttt{\string\YLcite} 命令,
    可以实现作者引用、已经带年份的作者引用。
    此外,注意 arXiv 文章自动使用 EB 类型,例如 \cite{kastner2018parallel},请使用 arXiv.org 直接导出 \texttt{@misc} 类型 bib
    而不是谷歌学术的 \texttt{@article} 类型 bib。
    arXiv 文章的访问时间(即 \texttt{urldate})均设置为 2024-03-01,
    如果需要修改可以在导言区添加:
    \begin{lstlisting}[language=tex, alsoletter={\\}, morekeywords={\\renewcommand,\\makeatletter,\\makeatother}]
\makeatletter
\renewcommand \seuthesis@arxiv@urldate {2024-04-01} % the date you want
\makeatother
    \end{lstlisting}

    所有使用效果如表~\ref{tab:cite_effect} 所示。
    \begin{xltabular}{\linewidth}{lX}
      \caption{引用效果示例} \label{tab:cite_effect} \\
      \longTableHdr{命令 & 效果 \\}
      \ttfamily 泰迪 \textcolor{Blue}{\string\cite}\{zhao2023ompl\} & 泰迪 \cite{zhao2023ompl} \\
      \ttfamily 泰迪 \textcolor{Blue}{\string\cite}[Fig.\string~1]\{zhao2023ompl\} & 泰迪 \cite[Fig.~1]{zhao2023ompl} \\
      \ttfamily 泰迪 \textcolor{Blue}{\string\cite}\{zhao2023ompl,you2023beam\} & 泰迪 \cite{zhao2023ompl,you2024beam} \\
      \ttfamily 泰迪 \textcolor{Blue}{\string\parencite}\{zhao2023ompl\} & 泰迪 \parencite{zhao2023ompl} \\
      \ttfamily \textcolor{Blue}{\string\Lcite}\{zhao2023ompl\} & \Lcite{zhao2023ompl} \\
      \ttfamily \textcolor{Blue}{\string\Lcite}[软件]\{zhao2023ompl\} & \Lcite[软件]{zhao2024mmcesim} \\
      \ttfamily \textcolor{Blue}{\string\Lcite}\{zhao2023ompl,you2023beam\} & \Lcite{zhao2023ompl,you2024beam} \\
      \ttfamily \textcolor{Blue}{\string\Lcite}\{zhao2023ompl,you2023beam\}[哈哈哈哈] & \Lcite{zhao2023ompl,you2024beam}[哈哈哈哈] \\
      \ttfamily \textcolor{Blue}{\string\Lcite}\{zhao2023ompl,you2023beam,\mbox{尤肖虎}20145g\} & \Lcite{zhao2023ompl,you2024beam,尤肖虎20145g} \\
      \ttfamily \textcolor{Blue}{\string\YLcite}\{zhao2023ompl\} & \YLcite{zhao2023ompl} \\
    \end{xltabular}

  \subsection{分类别示例}\label{subsec:bib_types}
    \textbf{期刊论文 [J]}:\parencite{zhao2023ompl,you2024beam,zhao2024flexible,尤肖虎20145g};
    \textbf{会议论文 [C]}:\parencite{zhao2023automatic};
    \textbf{计算机程序 [CP]}:\parencite{zhao2024mmcesim,zhao2024dg};
    \textbf{标准 [S]}:\parencite{IEEE802.11ad};
    \textbf{电子公告 [EB]}:\parencite{amd2023zcu111,kastner2018parallel};
    \textbf{普通图书 [M]}:\parencite{wong2017key};
    \textbf{学位论文 [D]}:\parencite{wipf2006bayesian};
    \textbf{报告 [R]}:\parencite{zhao2024dual}。% 此条目实际上按照 online 的格式处理

\section{章节}

  在 \LaTeX{} 中,
  章(一级标题)使用 \texttt{\string\chapter},
  节(二级标题)使用 \texttt{\string\section},
  小节(三级标题)使用 \texttt{\string\subsection},
  子小节(四级标题)使用 \texttt{\string\subsubsection},
  段落使用 \texttt{\string\paragraph}。
  官方 MS Word 模板仅提供第一、二级格式规范。
  二级标题及以上会出现在目录中,
  三级标题及以上会出现在 PDF 的 bookmarks 中。

  \subsection{三级标题}
    三级标题内容
    \subsubsection{四级标题}
    四级标题内容
      \paragraph{段落。}内容。


  \chapter{模板使用}\label{chap:usage}
  \section{文档编译}
模板编译需使用 XeLaTeX 引擎,其中参考文献需要使用 Biber 后端(\texttt{biblatex} 包)。
推荐使用 \texttt{latexmk} 工具编译,编译命令如下:
\begin{lstlisting}[morekeywords={latexmk}]
latexmk -pdfxe seuthesis2024b.tex
\end{lstlisting}

\section{文档选项}
文档选项设置详见表~\ref{tab:document_options}。

\begin{xltabular}{\linewidth}{ >{\ttfamily} l l >{\ttfamily} l X }
  \caption{\texttt{seuthesis2024b} 文档选项设置} \label{tab:document_options} \\
  \longTableHdr{\songti 文档选项 & \songti 类型 & \songti 默认值 & 介绍 \\}
  anonymous & bool & false & \textbf{匿名设置}。
    当论文需要匿名评审时,此选项设置为 \texttt{true}。
    相关命令包括 \texttt{\string\anony}。 \\
  bib titlecase & bool & false & \textbf{参考文献标题大小写设置}。
    与教务处提供的模板示例一致,默认使用 sentence case 而不是 title case。
    使用 title case 可将此选项设置为 \texttt{true}。 \\
  cnt in chapter & meta & \textrm{---} & \textbf{章节内编排设置}。
    这是默认设置,图表公式序号包括章节号,例如图~2-1,表~3.2,公式 (1.3)。
    相当于 \texttt{cnt in doc = false}。 \\
  cnt in doc & bool & false & \textbf{全文连续编排设置}。
    当论文较短,可以设置为 \texttt{true},图表公式的序号将按照全文连续编排,
    例如图~1,表~2,公式 (3)。 \\
  font dir & string & fonts/ & \textbf{字体文件夹}。
    设置字体文件夹路径,用于 \texttt{fontset = files} 时加载字体。
    (注意最后的斜杠 \texttt{/} 不可省略) \\
  fontset & string & files
    & \textbf{字体源设置}。
      可选值为 \texttt{files}(默认)、\texttt{mac~ms} 。
      \texttt{files} 使用 \texttt{font~dir} 选项设置的文件夹中字体,
      \texttt{mac~ms} 使用 macOS 上的 MS Word 字体。 \\
  my colors & bool & true
    & \textbf{导入一些 my 开头的颜色}。
      文档作者喜欢用的颜色组合,详见 \S\ref{sec:draw_color}。 \\
  no math & bool & false
    & \textbf{不使用数学设置}。
      不导入默认的数学设置,包括 \texttt{mathtools} 等宏包。 \\
  oneside & meta & \textrm{---}
    & \textbf{使用单页模式}。
      这是文档默认设置,等价于 \texttt{twoside = false}。 \\
  twoside & bool & false
    & \textbf{使用双页模式}。
      当论文页数很多需要使用双页打印时,此模式可以正确设置装订线位置和章开始页始终在正面。 \\
  showframe & bool & false
    & \textbf{显示页面框架}。
      用于调试页面布局,即使用 \texttt{geometry} 宏包的 \texttt{showframe} 选项。 \\
  use tex font & bool & false
    & \textbf{使用 \TeX{} 字体}。
      此选项将使用 Times New Roman 的克隆 TeX Gyre Terms 字体。 \\
  其他选项 & \textrm{---} & \textrm{---}
    & \texttt{seuthesis2024b.cls} 未定义的选项将自动作用于 \texttt{report} 基类。 \\
\end{xltabular}

\section{字体设置}
除了表~\ref{tab:font_effect} 中展示的组合外,
还有 \texttt{\string\songti}、\texttt{\string\heiti}、\texttt{\string\kaishu}、\texttt{\string\fangsong} 作为中文字体命令。
对于英文单词,还有 Small Capital 样式,使用 \texttt{\string\scshape} 切换,或者使用 \texttt{\string\textsc} 命令。
例如 \texttt{\string\textsc\{Matlab\}} 或 \texttt{\{\string\scshape\ Matlab\}} 会得到 \textsc{Matlab}。
注意 Times New Roman 需要新版字体,否则不包括 Small Capital 样式(例如 macOS 系统的 Times New Roman 就不是新版)。

字号设置可以使用 \href{https://ctan.org/pkg/ctex}{\texttt{ctex}} 宏包提供的 \texttt{\string\zihao} 命令,
例如 \texttt{\string\zihao\{5\}} 和 \texttt{\string\zihao\{-4\}} 分别表示五号和小四号字体。

\section{绘图与颜色}\label{sec:draw_color}
文档已自动导入 \href{https://ctan.org/pkg/tikz}{\texttt{tikz}}、\href{https://ctan.org/pkg/pgfplots}{\texttt{pgfplots}} 等宏包支持绘图。

文档导入 \href{https://ctan.org/pkg/xcolor}{\texttt{xcolor}} 宏包,并使用 \texttt{dvipsnames} 选项。
文档选项默认使用 \texttt{my colors} 选项,另外导入了一些作者喜欢的颜色,如图~\ref{fig:my_colors} 所示。
\begin{figure}[htbp]
  \caption{My Colors 中定义的颜色(RGB 为屏幕显示颜色、cmyk 为印刷颜色)}
  \label{fig:my_colors}
  \sffamily\footnotesize
  \begin{testcolors}[RGB,cmyk,hsb,gray]
    \testcolor{myblued}
    \testcolor{myred}
    \testcolor{mygreen}
    \testcolor{myyellow}
    \testcolor{mypurple}
    \testcolor{myblues}
    \testcolor{mypink}
  \end{testcolors}
\end{figure}

此外,\texttt{my colors} 选项也提供了 \texttt{sim} 的 \texttt{pgfplots} 曲线系列,
如图~\ref{fig:pgf_wave} 所示。
\begin{figure}[htbp]
  \caption{提供的 \texttt{sim} 曲线系列(供 pgfplots 使用)}
  \label{fig:pgf_wave}
  \begin{tikzpicture}
    \begin{axis}[
      , hide axis
      , xmin = 0
      , xmax = 10
      , ymin = 0
      , ymax = 10
      , legend entries = {序列1, 序列2, 序列3, 序列4, 序列5, 序列6, 序列7}
      , legend columns = -1
      , legend style = {/tikz/every even column/.append style = {column sep = 0.5cm, font = \sffamily}}
      , thick
    ]
      \addlegendimage{ myblued , thick, mark = o       , mark size = 2   }
      \addlegendimage{ myred   , thick, mark = diamond , mark size = 2.8 }
      \addlegendimage{ mygreen , thick, mark = square  , mark size = 2   }
      \addlegendimage{ myyellow, thick, mark = triangle, mark size = 3   }
      \addlegendimage{ mypurple, thick, mark = asterisk, mark size = 3   }
      \addlegendimage{ myblues , thick, mark = pentagon, mark size = 3   }
      \addlegendimage{ gray    , thick, mark = +       , mark size = 3   }
    \end{axis}
  \end{tikzpicture}  
\end{figure}

\section{其他命令}
\subsection{数字相关}
\subsubsection{罗马数字}
使用 \texttt{\string\Rn} 可以获得小写罗马数字,例如 \verb|\Rn{123}| 可以得到 \Rn{123}。
类似地,有 \texttt{\string\RN} 用于大写罗马数字,例如 \verb|\RN{1324}| 可以得到 \RN{1324}。
不过如果 I、II、III、IV 这类简单的可以直接字母输入。

\subsubsection{带圈数字}
使用 \texttt{\string\circNo} 可以获得带圈数字(从 \circNo{1}\ 到 \circNo{10}),
例如 \verb|\circNo{2}| 可以获得 \circNo{2}。
理论来说,可以使用 \texttt{\string\circNo} 生成从 0 到 50 的数字,但是只有 1 到 10 的数字有对应的中易宋体 Unicode 字符,
因此,其他任何输入均使用 \href{https://ctan.org/pkg/circledsteps}{\texttt{circledsteps}} 宏包的 \texttt{\string\CircleText} 基础上调整大小位置,效果不统一,
例如 \circNo{0}、\circNo{11}。
如果使用带星版本(\texttt{\string\circNo*})可以获得反白(实心带圈)数字,
例如 \verb|\circNo*{3}| 可以获得 \circNo*{3},
此时所有数字均使用 \href{https://ctan.org/pkg/circledsteps}{\texttt{circledsteps}} 宏包的 \texttt{\string\Circle} 生成。
当然,你也可以直接复制 Unicode 符号\ ①\ 等。

值得注意的是,\circNo{1}\ 到 \circNo{10}\ 被认为是中文字符,
因此与之后的中文字没有空格(如果你直接使用 Unicode 输入前后均没有空格),因此如果需要需要手动添加空格。
例如 {\color{Blue}\ \verb|第 \circNo{1} 名|} 可以得到第 \circNo{1} 名,
而 {\color{Blue}\ \verb|第 \circNo{1}\ 名|} 可以得到第 \circNo{1}\ 名。

一些数字的功能性测试:
带颜色 \textcolor{red}{\circNo{0} \circNo{1} \circNo{10} \circNo*{2} \circNo{11} \circNo*{99}};
上下标 1\textsuperscript{\circNo{0}}\ 2\textsuperscript{\circNo{1}}\ 3\textsubscript{\circNo*{11}}\ 4\textsubscript{\circNo{8}};
字体变化 \textsf{\circNo{0} \circNo{3} \circNo*{4}}(\texttt{\string\ttfamily})、
\textit{\circNo{0} \circNo{5} \circNo*{6}}(\texttt{\string\itshape},太糟糕了,不要这么干)、
\textbf{\circNo{0} \circNo{7} \circNo*{8}}(\texttt{\string\bfseries},也不太行)、
\texttt{\circNo{0} \circNo*{9} \circNo{10}}(\texttt{\string\ttfamily},还是不行)。

\section{杂项测试}
这一节不仅是一个测试,
也提供了一些你可以使用的环境例子。

\subsection{列表}
\texttt{itemize} (无序列表)环境测试:
\begin{itemize}
  \item 项目一
  \item Item 2
  \item 一个包含子项目的 item
  \begin{itemize}
    \item one
    \begin{enumerate}
      \item OMPL-SBL\cite{zhao2023ompl}
      \item Other works \dots
    \end{enumerate}
    \item two
    \item three
    \begin{itemize}
      \item A very very very very very very very very very very very very long sentence that spans multiple lines.
    \end{itemize}
  \end{itemize}
  \item Item 4
  \begin{enumerate}
    \item Teddy Bear
    \item Panda
  \end{enumerate}
\end{itemize}

\texttt{enumerate} (有序列表)环境测试:
\begin{enumerate}
  \item 另一个项目
  \item 再来一个
  \begin{enumerate}
    \item 二级项目
    \begin{enumerate}
      \item That is so boring.
    \end{enumerate}
    \item 另一个
    \item What about this?
    \begin{itemize}
      \item Hi! Visit \url{https://mmcesim.org} for more information.
      \item 太炫酷啦
    \end{itemize}
  \end{enumerate}
  \item 再来一个
\end{enumerate}

\subsection{数学公式}

带编号数学公式使用 \texttt{equation} 环境:
\begin{equation}\label{eq:demo:sum}
  x=\sum_{i=1}^{n}a_i.
\end{equation}

再来一个矩阵运算公式:
\begin{equation}
  \begin{bmatrix}
    U_{L1}\\I_{L1}
  \end{bmatrix}
  =
  \begin{bmatrix}
    0&jZ_{c1}\\jY_{c1}&0
  \end{bmatrix}
  \begin{bmatrix}
    0&jZ_{c2}\\jY_{c2}&0
  \end{bmatrix}
  \begin{bmatrix}
    U_{L2}\\I_{L2}
  \end{bmatrix}
  =
  \begin{bmatrix}
    -Z_{c1}Y_{c2}&0\\0&-Z_{c2}Y_{c1}
  \end{bmatrix}
  \begin{bmatrix}
    U_{L2}\\I_{L2}
  \end{bmatrix}.
\end{equation}

无编号数学环境使用 \texttt{equation*} 环境,或者 \texttt{\string\[ ... \string\]}:
\begin{equation*}
  \frac{U_2^2}{U_1^2}=\frac{Z_2}{\widetilde{Z}_2}=\left(\frac{Z_{c2}}{Z_{c1}}\right)^2
  \implies
  \frac{U_2}{U_1}=\left|\frac{U_{L2}}{U_{L1}}\right|=\frac{Z_{c2}}{Z_{c1}}.
\end{equation*}

子公式使用 \texttt{subequations} 环境:
\begin{subequations}\label{eq:demo:subeqs}
\begin{equation}\label{eq:demo:subeq1}
  A=B,
\end{equation}
\begin{equation}\label{eq:demo:subeq2}
  B=C,
\end{equation}
\end{subequations}
其中 $A$、$B$ 和 $C$ 是一些变量。
当然,你可以用 \texttt{aligned} 环境来改写公式 \eqref{eq:demo:subeqs}:
\begin{equation}
  \left\{
    \begin{aligned}
      A&=B,\\
      B&=C.
    \end{aligned}
  \right.
\end{equation}

使用 \texttt{\string\eqref} 引用公式,例如
公式 \eqref{eq:demo:sum},还有子公式 \eqref{eq:demo:subeq1} 和 \eqref{eq:demo:subeq2}。

\subsection{定理和证明}
定理设置使用了 \href{https://ctan.org/pkg/amsthm}{\texttt{amsthm}} 宏包,
可以自行定义定理环境。例如此文档导言区已设置:
\begin{lstlisting}[alsoletter={\\}, morekeywords={\\newtheorem}]
\newtheorem{theorem}{定理}[chapter]
\newtheorem{definition}{定义}[chapter]
\newtheorem{lemma}{引理}[chapter]
\newtheorem{corollary}{推论}[chapter]
\end{lstlisting}
其中,最后的选项 \texttt{[chapter]} 使得标号按章节编排。

一个定理的效果如下,同样可以使用 \texttt{\string\ref} 引用定理编号,
例如定理~\ref{thm:pythagorean}。
\begin{theorem}[勾股定理]\label{thm:pythagorean}
  直角三角形斜边的平方等于两直角边的平方和。
  The area of the square whose side is the hypotenuse (the side opposite the right angle)
  is equal to the sum of the areas of the squares on the other two sides.
\end{theorem}
证明使用 \texttt{proof} 环境。
\begin{proof}
  因为
  \begin{equation}
    1+1=2,
  \end{equation}
  所以 $1+1=2$。% 废话文学
\end{proof}

\begin{proof}[证明 Teddy 是一种熊]
  Proof 的名字也可以修改。
\end{proof}

\subsection{Ti\textit{k}Z 绘图}
一个 Ti\textit{k}Z 绘图如图~\ref{fig:tikz_wave} 所示。
\begin{figure}[htbp]
  \usetikzlibrary{positioning}
  \begin{tikzpicture}[
    , c node/.style = {
      , draw
      , circle
      , thick
      , inner sep = 0
      , outer sep = 0
      , line width = 0.5mm
      , minimum width = 1.5mm
    }
    , wire/.style = {
      line width = 1mm
    }
    , r node/.style = {
      , draw
      , very thick
      , fill = white
      , inner sep = 0.5mm
      , minimum height = 6mm
      , midway
    }
  ]
    \node (n-0) [c node] {};
    \node (n-1) [c node, left = 15mm of n-0.center] {};
    \node (n-2) [c node, left = 15mm of n-1.center] {};
    \node (n-3) [c node, left = 25mm of n-2.center] {};
    \foreach \x in {0, 1, 2, 3} {
      \node (m-\x) [c node, below = 16mm of n-\x.center] {};
    }
    \foreach \x in {0, 1, 2} {
      \node (a-\x) [c node, minimum width = 0.5mm, below = 6mm of m-\x.center] {};
    }
    \node [above = 1mm of n-2] {A};
    \draw [wire] (n-0) -- (n-1) -- (n-2) -- (n-3);
    \draw [wire] (m-0) -- (m-1) -- (m-2) -- (m-3);
    \draw (n-2) -- (m-2) node (Z1) [r node] {$Z_1$};
    \draw (n-0) -- ++ (4mm, 0) -- ++ (0, -16.75mm) node (Z2) [r node] {$Z_2$} -- (m-0);
    \node [right = 4mm of Z1] {$Z_{c1}$};
    \node [left = 6mm of Z1] {$Z_{c1}$};
    \node [left = 4mm of Z2] {$Z_{c2}$};
    \draw [-latex, thick] ([xshift = -25mm]a-2.center) -- (a-2) -- (a-0) -- ++ (7mm, 0) node [right] {$z$};
    \node [below = 1mm of a-0, font = \small] {$0$};
    \node [below = 1mm of a-1, font = \small] {$-\frac\lambda4$};
    \node [below = 1mm of a-2, font = \small] {$-\frac\lambda2$};
    \draw [->, red, thick] ([xshift = 2mm, yshift = -1mm]n-2.center) |- ++ (6mm, 3mm) node [above] {$\widetilde{Z}_2$};
    \draw [->, blue, thick] ([xshift = 1mm, yshift = -3mm]n-2.center) -- ++ (6mm, 0) node [right] {$\widetilde{I}_2$};
    \draw [->, blue, thick] ([xshift = 4mm, yshift = -5mm]n-2.center) -- ++ (0, -6mm) node [below] {$I_1$};
    \draw [->, blue, thick] ([xshift = 8mm, yshift = -5mm]n-0.center) -- ++ (0, -6mm) node [below] {$I_2$};
  \end{tikzpicture}%
  \caption{一个传输线 Ti\textit{k}Z 图}
  \label{fig:tikz_wave}
\end{figure}




  % 此处可以插入更多的章节

  % 参考文献
  \chapterBib\label{chap:bib}

  % 附录
  \appendix
  \chapter{附录名称}\label{chap:appendix}
  对于一些不宜放入正文中、但作为毕业设计(论文)又不可残缺的组成部分或具有重要参考价值的内容,
可编入毕业设计(论文)的附录中,例如,正文内过于冗长的公式推导、
方便他人阅读所需的辅助性数学工具或表格、重复性数据和图表、非常必要的程序说明和程序全文、关键调查问卷或方案等。

附录的格式与正文相同,如有多个附录需依顺序用大写字母A,B,C,……编序号,如附录A,附录B,附录C,……。
只有一个附录时也要编序号,即附录A。
每个附录应有标题,如:“附录A 参考文献著录规则及注意事项”。

附录一般与论文全文装订在一起,与正文一起编页码。

  
  % 致谢
  \chapterAck
  学位论文正文和附录之后,一般应放置致谢(后记或说明),主要感谢指导老师和对论文工作有直接贡献和帮助的人士和单位。致谢言语应谦虚诚恳,实事求是。字数一般不超过1000个汉字。

“致谢”用三号黑体加粗居中,两字之间空4个半角空格。致谢内容为小四号宋体,1.5倍行距。


\end{document}
