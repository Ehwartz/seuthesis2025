\section{文档编译}
模板编译需使用 XeLaTeX 引擎,其中参考文献需要使用 Biber 后端(\texttt{biblatex} 包)。
推荐使用 \texttt{latexmk} 工具编译,编译命令如下:
\begin{lstlisting}[morekeywords={latexmk}]
latexmk -pdfxe seuthesis2024b.tex
\end{lstlisting}

\section{文档选项}
文档选项设置详见表~\ref{tab:document_options}。

\begin{xltabular}{\linewidth}{ >{\ttfamily} l l >{\ttfamily} l X }
  \caption{\texttt{seuthesis2024b} 文档选项设置} \label{tab:document_options} \\
  \longTableHdr{\songti 文档选项 & \songti 类型 & \songti 默认值 & 介绍 \\}
  anonymous & bool & false & \textbf{匿名设置}。
    当论文需要匿名评审时,此选项设置为 \texttt{true}。
    相关命令包括 \texttt{\string\anony}。 \\
  bib titlecase & bool & false & \textbf{参考文献标题大小写设置}。
    与教务处提供的模板示例一致,默认使用 sentence case 而不是 title case。
    使用 title case 可将此选项设置为 \texttt{true}。 \\
  cnt in chapter & meta & \textrm{---} & \textbf{章节内编排设置}。
    这是默认设置,图表公式序号包括章节号,例如图~2-1,表~3.2,公式 (1.3)。
    相当于 \texttt{cnt in doc = false}。 \\
  cnt in doc & bool & false & \textbf{全文连续编排设置}。
    当论文较短,可以设置为 \texttt{true},图表公式的序号将按照全文连续编排,
    例如图~1,表~2,公式 (3)。 \\
  font dir & string & fonts/ & \textbf{字体文件夹}。
    设置字体文件夹路径,用于 \texttt{fontset = files} 时加载字体。
    (注意最后的斜杠 \texttt{/} 不可省略) \\
  fontset & string & files
    & \textbf{字体源设置}。
      可选值为 \texttt{files}(默认)、\texttt{mac~ms} 。
      \texttt{files} 使用 \texttt{font~dir} 选项设置的文件夹中字体,
      \texttt{mac~ms} 使用 macOS 上的 MS Word 字体。 \\
  my colors & bool & true
    & \textbf{导入一些 my 开头的颜色}。
      文档作者喜欢用的颜色组合,详见 \S\ref{sec:draw_color}。 \\
  no math & bool & false
    & \textbf{不使用数学设置}。
      不导入默认的数学设置,包括 \texttt{mathtools} 等宏包。 \\
  oneside & meta & \textrm{---}
    & \textbf{使用单页模式}。
      这是文档默认设置,等价于 \texttt{twoside = false}。 \\
  twoside & bool & false
    & \textbf{使用双页模式}。
      当论文页数很多需要使用双页打印时,此模式可以正确设置装订线位置和章开始页始终在正面。 \\
  showframe & bool & false
    & \textbf{显示页面框架}。
      用于调试页面布局,即使用 \texttt{geometry} 宏包的 \texttt{showframe} 选项。 \\
  use tex font & bool & false
    & \textbf{使用 \TeX{} 字体}。
      此选项将使用 Times New Roman 的克隆 TeX Gyre Terms 字体。 \\
  其他选项 & \textrm{---} & \textrm{---}
    & \texttt{seuthesis2024b.cls} 未定义的选项将自动作用于 \texttt{report} 基类。 \\
\end{xltabular}

\section{字体设置}
除了表~\ref{tab:font_effect} 中展示的组合外,
还有 \texttt{\string\songti}、\texttt{\string\heiti}、\texttt{\string\kaishu}、\texttt{\string\fangsong} 作为中文字体命令。
对于英文单词,还有 Small Capital 样式,使用 \texttt{\string\scshape} 切换,或者使用 \texttt{\string\textsc} 命令。
例如 \texttt{\string\textsc\{Matlab\}} 或 \texttt{\{\string\scshape\ Matlab\}} 会得到 \textsc{Matlab}。
注意 Times New Roman 需要新版字体,否则不包括 Small Capital 样式(例如 macOS 系统的 Times New Roman 就不是新版)。

字号设置可以使用 \href{https://ctan.org/pkg/ctex}{\texttt{ctex}} 宏包提供的 \texttt{\string\zihao} 命令,
例如 \texttt{\string\zihao\{5\}} 和 \texttt{\string\zihao\{-4\}} 分别表示五号和小四号字体。

\section{绘图与颜色}\label{sec:draw_color}
文档已自动导入 \href{https://ctan.org/pkg/tikz}{\texttt{tikz}}、\href{https://ctan.org/pkg/pgfplots}{\texttt{pgfplots}} 等宏包支持绘图。

文档导入 \href{https://ctan.org/pkg/xcolor}{\texttt{xcolor}} 宏包,并使用 \texttt{dvipsnames} 选项。
文档选项默认使用 \texttt{my colors} 选项,另外导入了一些作者喜欢的颜色,如图~\ref{fig:my_colors} 所示。
\begin{figure}[htbp]
  \caption{My Colors 中定义的颜色(RGB 为屏幕显示颜色、cmyk 为印刷颜色)}
  \label{fig:my_colors}
  \sffamily\footnotesize
  \begin{testcolors}[RGB,cmyk,hsb,gray]
    \testcolor{myblued}
    \testcolor{myred}
    \testcolor{mygreen}
    \testcolor{myyellow}
    \testcolor{mypurple}
    \testcolor{myblues}
    \testcolor{mypink}
  \end{testcolors}
\end{figure}

此外,\texttt{my colors} 选项也提供了 \texttt{sim} 的 \texttt{pgfplots} 曲线系列,
如图~\ref{fig:pgf_wave} 所示。
\begin{figure}[htbp]
  \caption{提供的 \texttt{sim} 曲线系列(供 pgfplots 使用)}
  \label{fig:pgf_wave}
  \begin{tikzpicture}
    \begin{axis}[
      , hide axis
      , xmin = 0
      , xmax = 10
      , ymin = 0
      , ymax = 10
      , legend entries = {序列1, 序列2, 序列3, 序列4, 序列5, 序列6, 序列7}
      , legend columns = -1
      , legend style = {/tikz/every even column/.append style = {column sep = 0.5cm, font = \sffamily}}
      , thick
    ]
      \addlegendimage{ myblued , thick, mark = o       , mark size = 2   }
      \addlegendimage{ myred   , thick, mark = diamond , mark size = 2.8 }
      \addlegendimage{ mygreen , thick, mark = square  , mark size = 2   }
      \addlegendimage{ myyellow, thick, mark = triangle, mark size = 3   }
      \addlegendimage{ mypurple, thick, mark = asterisk, mark size = 3   }
      \addlegendimage{ myblues , thick, mark = pentagon, mark size = 3   }
      \addlegendimage{ gray    , thick, mark = +       , mark size = 3   }
    \end{axis}
  \end{tikzpicture}  
\end{figure}

\section{其他命令}
\subsection{数字相关}
\subsubsection{罗马数字}
使用 \texttt{\string\Rn} 可以获得小写罗马数字,例如 \verb|\Rn{123}| 可以得到 \Rn{123}。
类似地,有 \texttt{\string\RN} 用于大写罗马数字,例如 \verb|\RN{1324}| 可以得到 \RN{1324}。
不过如果 I、II、III、IV 这类简单的可以直接字母输入。

\subsubsection{带圈数字}
使用 \texttt{\string\circNo} 可以获得带圈数字(从 \circNo{1}\ 到 \circNo{10}),
例如 \verb|\circNo{2}| 可以获得 \circNo{2}。
理论来说,可以使用 \texttt{\string\circNo} 生成从 0 到 50 的数字,但是只有 1 到 10 的数字有对应的中易宋体 Unicode 字符,
因此,其他任何输入均使用 \href{https://ctan.org/pkg/circledsteps}{\texttt{circledsteps}} 宏包的 \texttt{\string\CircleText} 基础上调整大小位置,效果不统一,
例如 \circNo{0}、\circNo{11}。
如果使用带星版本(\texttt{\string\circNo*})可以获得反白(实心带圈)数字,
例如 \verb|\circNo*{3}| 可以获得 \circNo*{3},
此时所有数字均使用 \href{https://ctan.org/pkg/circledsteps}{\texttt{circledsteps}} 宏包的 \texttt{\string\Circle} 生成。
当然,你也可以直接复制 Unicode 符号\ ①\ 等。

值得注意的是,\circNo{1}\ 到 \circNo{10}\ 被认为是中文字符,
因此与之后的中文字没有空格(如果你直接使用 Unicode 输入前后均没有空格),因此如果需要需要手动添加空格。
例如 {\color{Blue}\ \verb|第 \circNo{1} 名|} 可以得到第 \circNo{1} 名,
而 {\color{Blue}\ \verb|第 \circNo{1}\ 名|} 可以得到第 \circNo{1}\ 名。

一些数字的功能性测试:
带颜色 \textcolor{red}{\circNo{0} \circNo{1} \circNo{10} \circNo*{2} \circNo{11} \circNo*{99}};
上下标 1\textsuperscript{\circNo{0}}\ 2\textsuperscript{\circNo{1}}\ 3\textsubscript{\circNo*{11}}\ 4\textsubscript{\circNo{8}};
字体变化 \textsf{\circNo{0} \circNo{3} \circNo*{4}}(\texttt{\string\ttfamily})、
\textit{\circNo{0} \circNo{5} \circNo*{6}}(\texttt{\string\itshape},太糟糕了,不要这么干)、
\textbf{\circNo{0} \circNo{7} \circNo*{8}}(\texttt{\string\bfseries},也不太行)、
\texttt{\circNo{0} \circNo*{9} \circNo{10}}(\texttt{\string\ttfamily},还是不行)。

\section{杂项测试}
这一节不仅是一个测试,
也提供了一些你可以使用的环境例子。

\subsection{列表}
\texttt{itemize} (无序列表)环境测试:
\begin{itemize}
  \item 项目一
  \item Item 2
  \item 一个包含子项目的 item
  \begin{itemize}
    \item one
    \begin{enumerate}
      \item OMPL-SBL\cite{zhao2023ompl}
      \item Other works \dots
    \end{enumerate}
    \item two
    \item three
    \begin{itemize}
      \item A very very very very very very very very very very very very long sentence that spans multiple lines.
    \end{itemize}
  \end{itemize}
  \item Item 4
  \begin{enumerate}
    \item Teddy Bear
    \item Panda
  \end{enumerate}
\end{itemize}

\texttt{enumerate} (有序列表)环境测试:
\begin{enumerate}
  \item 另一个项目
  \item 再来一个
  \begin{enumerate}
    \item 二级项目
    \begin{enumerate}
      \item That is so boring.
    \end{enumerate}
    \item 另一个
    \item What about this?
    \begin{itemize}
      \item Hi! Visit \url{https://mmcesim.org} for more information.
      \item 太炫酷啦
    \end{itemize}
  \end{enumerate}
  \item 再来一个
\end{enumerate}

\subsection{数学公式}

带编号数学公式使用 \texttt{equation} 环境:
\begin{equation}\label{eq:demo:sum}
  x=\sum_{i=1}^{n}a_i.
\end{equation}

再来一个矩阵运算公式:
\begin{equation}
  \begin{bmatrix}
    U_{L1}\\I_{L1}
  \end{bmatrix}
  =
  \begin{bmatrix}
    0&jZ_{c1}\\jY_{c1}&0
  \end{bmatrix}
  \begin{bmatrix}
    0&jZ_{c2}\\jY_{c2}&0
  \end{bmatrix}
  \begin{bmatrix}
    U_{L2}\\I_{L2}
  \end{bmatrix}
  =
  \begin{bmatrix}
    -Z_{c1}Y_{c2}&0\\0&-Z_{c2}Y_{c1}
  \end{bmatrix}
  \begin{bmatrix}
    U_{L2}\\I_{L2}
  \end{bmatrix}.
\end{equation}

无编号数学环境使用 \texttt{equation*} 环境,或者 \texttt{\string\[ ... \string\]}:
\begin{equation*}
  \frac{U_2^2}{U_1^2}=\frac{Z_2}{\widetilde{Z}_2}=\left(\frac{Z_{c2}}{Z_{c1}}\right)^2
  \implies
  \frac{U_2}{U_1}=\left|\frac{U_{L2}}{U_{L1}}\right|=\frac{Z_{c2}}{Z_{c1}}.
\end{equation*}

子公式使用 \texttt{subequations} 环境:
\begin{subequations}\label{eq:demo:subeqs}
\begin{equation}\label{eq:demo:subeq1}
  A=B,
\end{equation}
\begin{equation}\label{eq:demo:subeq2}
  B=C,
\end{equation}
\end{subequations}
其中 $A$、$B$ 和 $C$ 是一些变量。
当然,你可以用 \texttt{aligned} 环境来改写公式 \eqref{eq:demo:subeqs}:
\begin{equation}
  \left\{
    \begin{aligned}
      A&=B,\\
      B&=C.
    \end{aligned}
  \right.
\end{equation}

使用 \texttt{\string\eqref} 引用公式,例如
公式 \eqref{eq:demo:sum},还有子公式 \eqref{eq:demo:subeq1} 和 \eqref{eq:demo:subeq2}。

\subsection{定理和证明}
定理设置使用了 \href{https://ctan.org/pkg/amsthm}{\texttt{amsthm}} 宏包,
可以自行定义定理环境。例如此文档导言区已设置:
\begin{lstlisting}[alsoletter={\\}, morekeywords={\\newtheorem}]
\newtheorem{theorem}{定理}[chapter]
\newtheorem{definition}{定义}[chapter]
\newtheorem{lemma}{引理}[chapter]
\newtheorem{corollary}{推论}[chapter]
\end{lstlisting}
其中,最后的选项 \texttt{[chapter]} 使得标号按章节编排。

一个定理的效果如下,同样可以使用 \texttt{\string\ref} 引用定理编号,
例如定理~\ref{thm:pythagorean}。
\begin{theorem}[勾股定理]\label{thm:pythagorean}
  直角三角形斜边的平方等于两直角边的平方和。
  The area of the square whose side is the hypotenuse (the side opposite the right angle)
  is equal to the sum of the areas of the squares on the other two sides.
\end{theorem}
证明使用 \texttt{proof} 环境。
\begin{proof}
  因为
  \begin{equation}
    1+1=2,
  \end{equation}
  所以 $1+1=2$。% 废话文学
\end{proof}

\begin{proof}[证明 Teddy 是一种熊]
  Proof 的名字也可以修改。
\end{proof}

\subsection{Ti\textit{k}Z 绘图}
一个 Ti\textit{k}Z 绘图如图~\ref{fig:tikz_wave} 所示。
\begin{figure}[htbp]
  \usetikzlibrary{positioning}
  \begin{tikzpicture}[
    , c node/.style = {
      , draw
      , circle
      , thick
      , inner sep = 0
      , outer sep = 0
      , line width = 0.5mm
      , minimum width = 1.5mm
    }
    , wire/.style = {
      line width = 1mm
    }
    , r node/.style = {
      , draw
      , very thick
      , fill = white
      , inner sep = 0.5mm
      , minimum height = 6mm
      , midway
    }
  ]
    \node (n-0) [c node] {};
    \node (n-1) [c node, left = 15mm of n-0.center] {};
    \node (n-2) [c node, left = 15mm of n-1.center] {};
    \node (n-3) [c node, left = 25mm of n-2.center] {};
    \foreach \x in {0, 1, 2, 3} {
      \node (m-\x) [c node, below = 16mm of n-\x.center] {};
    }
    \foreach \x in {0, 1, 2} {
      \node (a-\x) [c node, minimum width = 0.5mm, below = 6mm of m-\x.center] {};
    }
    \node [above = 1mm of n-2] {A};
    \draw [wire] (n-0) -- (n-1) -- (n-2) -- (n-3);
    \draw [wire] (m-0) -- (m-1) -- (m-2) -- (m-3);
    \draw (n-2) -- (m-2) node (Z1) [r node] {$Z_1$};
    \draw (n-0) -- ++ (4mm, 0) -- ++ (0, -16.75mm) node (Z2) [r node] {$Z_2$} -- (m-0);
    \node [right = 4mm of Z1] {$Z_{c1}$};
    \node [left = 6mm of Z1] {$Z_{c1}$};
    \node [left = 4mm of Z2] {$Z_{c2}$};
    \draw [-latex, thick] ([xshift = -25mm]a-2.center) -- (a-2) -- (a-0) -- ++ (7mm, 0) node [right] {$z$};
    \node [below = 1mm of a-0, font = \small] {$0$};
    \node [below = 1mm of a-1, font = \small] {$-\frac\lambda4$};
    \node [below = 1mm of a-2, font = \small] {$-\frac\lambda2$};
    \draw [->, red, thick] ([xshift = 2mm, yshift = -1mm]n-2.center) |- ++ (6mm, 3mm) node [above] {$\widetilde{Z}_2$};
    \draw [->, blue, thick] ([xshift = 1mm, yshift = -3mm]n-2.center) -- ++ (6mm, 0) node [right] {$\widetilde{I}_2$};
    \draw [->, blue, thick] ([xshift = 4mm, yshift = -5mm]n-2.center) -- ++ (0, -6mm) node [below] {$I_1$};
    \draw [->, blue, thick] ([xshift = 8mm, yshift = -5mm]n-0.center) -- ++ (0, -6mm) node [below] {$I_2$};
  \end{tikzpicture}%
  \caption{一个传输线 Ti\textit{k}Z 图}
  \label{fig:tikz_wave}
\end{figure}


