具体研究内容每一章应另起页书写书写,层次要清楚,内容要有逻辑性,每一章标题需要按论文实际研究内容进行填写,不可直接写成第二章 正文。研究内容因学科、选题特点可有差异,但必须言之成理,论据可靠,严格遵循本学科国际通行的学术规范。

中文为小四号宋体,英文及数字为小四号Times New Roman,首行缩进2个字符,行间距为1.5倍。

\section{插图格式要求}

  插图力求精炼,且每个插图均应有图序和图名。图序与图名位于插图下方,图序一般按章节编排,如图1-1(第一章第1个图),在插图较少时可以全文连续编序,如图10。
  
  如一个插图由两个及以上的分图组成,分图用(a)、(b)、(c)等标出,并标出分图名。

  简单文字图可用 \LaTeX{} 自带宏包 Ti\textit{k}Z 直接绘制,
  复杂的图考虑使用 standalone 的 Ti\textit{k}Z 完成,以提高图形表达质量。

  插图居中排列,与上文文本之间空一行。图序图名设置为五号宋体居中,图序与图名之间空一格。
  可爱的图~\ref{fig:duck}。
  \begin{figure}[htbp]
    \includegraphics{example-image-duck}
    \caption{经典 \LaTeX{} 鸭子}
    \label{fig:duck}
  \end{figure}

  图~\ref{fig:subfigs} 包含两个子图:
  图~\subref*{subfig:s1} 和 \subref*{subfig:s2}。
  \begin{figure}[htbp]
    \subfloat[子图一\label{subfig:s1}]{\includegraphics[width=.3\linewidth]{example-image-a}}\quad
    \subfloat[子图二\label{subfig:s2}]{\includegraphics[width=.3\linewidth]{example-image-b}}
    \caption{两个子图}
    \label{fig:subfigs}
  \end{figure}

  图序章节编排为默认,全文连续编排需设置文档选项 \texttt{cnt in doc}。此设置同时作用于表格和公式。

\section{表格格式要求}

  表格的结构应简洁,一律采用三线表,应有表序和表名,且表序和表名位于表格上方。
  表格可以逐章单独编序(如:表2.1),也可以统一编序(如:表10),采用哪种方式应和插图及公式的编序方式统一。表序必须连续,不得重复或跳跃。

  表格无法在同一页排版时,可以用续表的形式另页书写,续表需在表格右上角表序前加“续”字,如“续表2.1”,并重复表头。

  表格居中,边框为黑色直线1磅,中文为五号宋体,英文及数字为五号Times New Roman字体,表序与表名之间空一格,表格与下文之间空一行。

  表格例子如表~\ref{tab:font_effect} 所示。
  表格内字号默认为五号(10.5pt),即 \texttt{\string\small} 或 \texttt{\string\zihao\{5\}}。

  \begin{table}[htbp]
    \caption{字体族、字体形状和字体系列的组合效果}
    \label{tab:font_effect}
    \begin{tabular}{ccccc}
      \toprule
      & & \verb|\itshape| & \verb|\bfseries| & \verb|\itshape\bfseries| \\
      \midrule
      \verb|\rmfamily| & \rmfamily 罗马体 roman & \rmfamily\itshape 倾斜 italic & \rmfamily\bfseries 加粗 bold & \rmfamily\itshape\bfseries 粗斜 bold-italic \\
      \verb|\sffamily| & \sffamily 无衬线 sans  & \sffamily\itshape 倾斜 italic & \sffamily\bfseries 加粗 bold & \sffamily\itshape\bfseries 粗斜 bold-italic \\
      \verb|\ttfamily| & \ttfamily 打字机 mono  & \ttfamily\itshape 倾斜 italic & \ttfamily\bfseries 加粗 bold & \ttfamily\itshape\bfseries 粗斜 bold-italic \\
      \bottomrule
    \end{tabular}
  \end{table}

  跨页表格可以使用 \texttt{longtable} 环境。
  但是为了更好的表格居中效果,推荐使用 \texttt{xltabular} 环境,
  将 \texttt{longtable} 与 \texttt{tabularx} 结合。
  效果如表~\ref{tab:arabic_roman}。

  \begin{xltabular}{\linewidth}{YYYY} % <-- centered X columns
    \caption{阿拉伯数字与罗马数字转换表} \label{tab:arabic_roman} \\
    \longTableHdr{阿拉伯数字 & 罗马数字 & 阿拉伯数字 & 罗马数字 \\}
    1 & \RN{1} & 2 & \RN{2} \\
    3 & \RN{3} & 4 & \RN{4} \\
    5 & \RN{5} & 6 & \RN{6} \\
    7 & \RN{7} & 8 & \RN{8} \\
    9 & \RN{9} & 10 & \RN{10} \\
    11 & \RN{11} & 12 & \RN{12} \\
    13 & \RN{13} & 14 & \RN{14} \\
    15 & \RN{15} & 16 & \RN{16} \\
    17 & \RN{17} & 18 & \RN{18} \\
    19 & \RN{19} & 20 & \RN{20} \\
    50 & \RN{50} & 100 & \RN{100} \\
    500 & \RN{500} & 2024 & \RN{2024} \\
  \end{xltabular}

\section{表达式}

  论文中的公式应注序号并加圆括号,序号一律用阿拉伯数字连续编序(如 (28) )或逐章编序(如 (3.6) ),
  编号方式应与插图、表格方式一致。序号排在版面右侧,且距右边距离相等。公式与序号之间不加虚线。

  长公式在一行无法写完的情况下,原则上应在等号(或数学符号,如“$+$”、“$-$”号)处换行,数学符号在换行的行首。

  公式及文字中的一般变量(或一般函数)(如坐标$X$、$Y$,电压$V$,频率$f$)宜用斜体,矢量用粗斜体如$\bm{S}$或白斜体上加单箭头,
  常用函数(如三角函数$\cos$、对数函数$\log$等)、数字运算符、化学元素符号及分子式、单位符号、产品代号、人名地名的外文字母等用正体。

  公式排版直接使用 \texttt{equation} 环境,如公式~\eqref{eq:einstein}。
  \begin{equation}\label{eq:einstein}
    E=mc^2.
  \end{equation}

\section{注释}

  正文中有个别名词或情况需要解释时,可加注说明,注释采用页末注(将注文放在加注页的下端)。
  在引文的右上角标注序号 \circNo{1}、\circNo{2}、……,如“马尔可夫链\footnote{马尔可夫链表示……}”。若在同一页中有两个以上的注时,按各注出现的先后,顺序编号。
  引文序号,以页为单位,且注释只限于写在注释符号出现的同页,不得隔页。

  注释采用小五号宋体,英文及数字为小五号Times New Roman字体,利用 \texttt{\string\footnote} 插入。

\section{参考文献}

  \subsection{引用格式}
    列出作者直接阅读过或在正文中引用过的文献资料。撰写论文时,需注意引用权威和最新的文献。

    参考文献需在引文右上角用方括号“[]”标明序号,如“基本机构\textsuperscript{[1]}”,并在参考文献中列出。
    每一条参考文献著录均以“.”结束。参考文献要另起一页,一律放在正文之后,不得放在各章节之后。

    参考文献采用顺序编码制,需符合《信息与文献 参考文献著录规则》(GB/T 7714-2015)规范要求,文献类型和标识代码为:
    普通图书[M]、会议录[C]、汇编[G]、报纸[N]、期刊[J]、学位论文[D]、报告[R]、标准[S]、专利[P]、数据库[DB]、
    计算机程序[CP]、电子公告[EB]、档案[A]、舆图[CM]、数据集[DS]、其他[Z]。

    参考文献中主要责任者的个人作者采用姓在前名在后的著录形式,当作者不超过3个时,全部照录。
    超过3个,著录的前3个作者其后加“,等”(, et al)。
    欧美著者的名可用缩写字母,缩写名后省略缩写点,姓和缩写名全大写。
    用汉语拼音书写的人名,姓全大写,名可缩写,取每个汉字拼音的首字母。

    参考文献为五号宋体,英文及数字为五号Times New Roman字体,两端对齐。
    参考文献中的标点符号均为英文标点,常用的参考文献著录项目和格式见\hyperref[chap:bib]{参考文献}%
    和第~\ref{subsec:bib_types} 节。

  \subsection{使用示例}
    算法方面的研究包括:
    OMPL-SBL 算法\cite{zhao2023ompl},
    RIS 的波束设计方法\cite{you2024beam};
    硬件设计方面的研究包括:
    高层级综合工具 FLAMES 库设计\cite{zhao2024flexible},
    自动生成语言 AHDW\cite{zhao2023automatic}。
    \name 的所有工作有\cite{zhao2023ompl,you2024beam,zhao2024flexible,zhao2023automatic}。

    使用 \texttt{\string\cite} 的引用均为上标;
    如果需要非上标,可以使用 \texttt{\string\parencite},例如 \parencite{zhao2023ompl}。
    此模版额外提供了 \texttt{\string\Lcite} 和 \texttt{\string\YLcite} 命令,
    可以实现作者引用、已经带年份的作者引用。
    此外,注意 arXiv 文章自动使用 EB 类型,例如 \cite{kastner2018parallel},请使用 arXiv.org 直接导出 \texttt{@misc} 类型 bib
    而不是谷歌学术的 \texttt{@article} 类型 bib。
    arXiv 文章的访问时间(即 \texttt{urldate})均设置为 2024-03-01,
    如果需要修改可以在导言区添加:
    \begin{lstlisting}[language=tex, alsoletter={\\}, morekeywords={\\renewcommand,\\makeatletter,\\makeatother}]
\makeatletter
\renewcommand \seuthesis@arxiv@urldate {2024-04-01} % the date you want
\makeatother
    \end{lstlisting}

    所有使用效果如表~\ref{tab:cite_effect} 所示。
    \begin{xltabular}{\linewidth}{lX}
      \caption{引用效果示例} \label{tab:cite_effect} \\
      \longTableHdr{命令 & 效果 \\}
      \ttfamily 泰迪 \textcolor{Blue}{\string\cite}\{zhao2023ompl\} & 泰迪 \cite{zhao2023ompl} \\
      \ttfamily 泰迪 \textcolor{Blue}{\string\cite}[Fig.\string~1]\{zhao2023ompl\} & 泰迪 \cite[Fig.~1]{zhao2023ompl} \\
      \ttfamily 泰迪 \textcolor{Blue}{\string\cite}\{zhao2023ompl,you2023beam\} & 泰迪 \cite{zhao2023ompl,you2024beam} \\
      \ttfamily 泰迪 \textcolor{Blue}{\string\parencite}\{zhao2023ompl\} & 泰迪 \parencite{zhao2023ompl} \\
      \ttfamily \textcolor{Blue}{\string\Lcite}\{zhao2023ompl\} & \Lcite{zhao2023ompl} \\
      \ttfamily \textcolor{Blue}{\string\Lcite}[软件]\{zhao2023ompl\} & \Lcite[软件]{zhao2024mmcesim} \\
      \ttfamily \textcolor{Blue}{\string\Lcite}\{zhao2023ompl,you2023beam\} & \Lcite{zhao2023ompl,you2024beam} \\
      \ttfamily \textcolor{Blue}{\string\Lcite}\{zhao2023ompl,you2023beam\}[哈哈哈哈] & \Lcite{zhao2023ompl,you2024beam}[哈哈哈哈] \\
      \ttfamily \textcolor{Blue}{\string\Lcite}\{zhao2023ompl,you2023beam,\mbox{尤肖虎}20145g\} & \Lcite{zhao2023ompl,you2024beam,尤肖虎20145g} \\
      \ttfamily \textcolor{Blue}{\string\YLcite}\{zhao2023ompl\} & \YLcite{zhao2023ompl} \\
    \end{xltabular}

  \subsection{分类别示例}\label{subsec:bib_types}
    \textbf{期刊论文 [J]}:\parencite{zhao2023ompl,you2024beam,zhao2024flexible,尤肖虎20145g};
    \textbf{会议论文 [C]}:\parencite{zhao2023automatic};
    \textbf{计算机程序 [CP]}:\parencite{zhao2024mmcesim,zhao2024dg};
    \textbf{标准 [S]}:\parencite{IEEE802.11ad};
    \textbf{电子公告 [EB]}:\parencite{amd2023zcu111,kastner2018parallel};
    \textbf{普通图书 [M]}:\parencite{wong2017key};
    \textbf{学位论文 [D]}:\parencite{wipf2006bayesian};
    \textbf{报告 [R]}:\parencite{zhao2024dual}。% 此条目实际上按照 online 的格式处理

\section{章节}

  在 \LaTeX{} 中,
  章(一级标题)使用 \texttt{\string\chapter},
  节(二级标题)使用 \texttt{\string\section},
  小节(三级标题)使用 \texttt{\string\subsection},
  子小节(四级标题)使用 \texttt{\string\subsubsection},
  段落使用 \texttt{\string\paragraph}。
  官方 MS Word 模板仅提供第一、二级格式规范。
  二级标题及以上会出现在目录中,
  三级标题及以上会出现在 PDF 的 bookmarks 中。

  \subsection{三级标题}
    三级标题内容
    \subsubsection{四级标题}
    四级标题内容
      \paragraph{段落。}内容。
